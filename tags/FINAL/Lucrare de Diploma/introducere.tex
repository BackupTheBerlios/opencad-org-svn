\chapter{Introducere}

OpenCad.org este un proiect iniţiat de autor în anul 2005 cu gîndul de a deveni 
o unealtă de real folos în activitatea de proiectare în domeniul arhitecturii.

Pornit de la o pasiune personală pentru domeniu, ideea acestui proiect vine ca 
o reacţie a curentului open-source, în care din ce în ce mai mulţi programatori 
vor să încerce să ofere soluţii performante şi în acelaşi timp gratuite. Deşi 
această primă determinare a proiectului nu a fost hotărîtoare, desigur eternele 
aventuri în lumea proiectării 3D ale vieţii pre-universitare au avut un rol 
decisiv în hotărîrea temei pentru acest proiect de diplomă.

Proiectul nostru are un slogan, care traducerea în engleză a sintagmei 
``Construieşte, nu desena". El reprezintă tema acestui proiect, anume izolarea 
activităţii creative de proiectarea a spaţiilor de locuit sau a clădirilor cu 
diverse scopuri de activităţile de desenare, de cunoştinţele stricte de desen 
tehnic, de teoria proiecţiilor tridimensionale şi a altor sarcini care stau în 
calea procesului creativ al unui arhitect.

Măcar în viziunea noastră, un proiectant poate să se elibereze de problemele 
desenului şi să se concentreze asupra produsului final, construcţia, prin 
intermediul unor unelte similare cu cele puse la dispoziţie de acest proiect.

Noi nu am adus ceva nou, nu am inventat o nouă metodă de proiectare. Ci am 
încercat să aducem o nouă perspectivă şi o nouă formă de lucru în activitatea 
de proiectare, cu o unealtă integrată cu unul dintre cele mai populare medii de 
dezvoltare actuale, Eclipse IDE.

Dacă OpenCad.org va deveni vreodată ceea se doreşte a fi, asta înseamnă că 
măcar un om sau un proiect a beneficiat de această muncă despre care vom 
discuta în următoarele pagini. De la faza de proiectare pînă la cele mai mici 
detalii de implementare, s-a avut întotdeauna în vedere scopul final al 
proiectului şi impactul asupra utilizatorului.

Credem că OpenCad.org este un început bun pentru ceea ce poate deveni un 
proiect de succes pentru lumea open-source cît şi pentru domeniul proiectării 
asistate de calculator în arhitectură. Există puţine astfel de iniţiative în 
această ramură tehnică, şi cu atît mai puţine, din păcate, originare şcolii 
româneşti de studii în domeniul calculatoarelor. Sperăm măcar într-o nouă 
deschidere spre viitor a cît mai multe iniţiative orientate atît către produse 
comerciale de mare anvergură cît şi proiecte open-source construite pe baza 
unei comunităţi de dezvoltatori.

Documentul de faţă este construit pentru a aduce cititorului o idee generală 
despre natura aplicaţiei noastre, a tehnologiilor implicate şi a unor detalii 
de implementare.

Vom trece în revistă unele aspecte relevante asupra dezvoltării acestui proiect 
şi vom încerca să tragem o concluzie asupra realizărilor prezentate aici şi 
asupra paşilor ce trebuie urmaţi pentru continuarea dezvoltării acestei 
aplicaţii.