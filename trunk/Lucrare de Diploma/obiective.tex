\chapter{Obiective}

Dintotdeauna am considerat că înainte de a iniţia orice activitate, fie că ea
este dezvoltarea unui produs software, fie că este orice altfel de sarcină de
amploare, autorul trebuie să-şi pună pe hîrtie o suită limitată de ţinte, nu
foarte uşor de realizat însă nu imposibile.

Pentru proiectul OpenCad.org, ele au fost alese prin perspectiva altor proiecte
similare dezvoltate în acest domeniu, şi am încercat să oferim ceva nou,
atractiv pentru eventualii utilizatori.

\section*{Uşurinţă în utilizare}

Dacă orice aplicaţie CAD tinde să ofere utilizatorilor ei unelte cît mai
puternice de dezvoltare, de multe ori se pierde din vedere ergonomia aplicaţiei.
Există un compromis ce trebuie făcut între capabilităţile aplicaţiei şi
ergonomie, însă nu credem că a cădea în oricare din tabere este benefic
rezultatului final al produsului.

De aceea, în realizarea acestui proiect am avut întotdeauna în vedere
interacţiunea cu utilizatorul şi interfaţa programului. Naturaleţea şi intuiţia
sunt cele mai bune metode de învăţare, şi nici o aplicaţie nu ar trebui să se
rezume la manualul sau documentaţia sa pentru a permite utilizatorilor să
înceapă să folosească această aplicaţie.

Noi ne-am dorit ca un utilizator să fie capabil să vadă rezultate efective
utilizînd aplicaţia noastră în cîteva minute după ce a deschis pentru prima dată
programul. Există multe aplicaţii CAD care necesită ore, dacă nu zile întregi de
acomodare cu setul de comenzi, setul de capabilităţi, de proprietăţi şi toate
celelalte componente ale unei aplicaţii de mare anvergură.

\section*{Puternică unealtă de dezvoltare}

Desigur, nu puteam ignora scopul final al activităţii de proiectare, şi anume
redarea cît mai fidelă a ideilor proiectantului. Dacă printr-o ergonomie
defectuasă acest deziderat poate fi împedicat prin încetinirea procesului, lipsa
totală a anumitor capacităţi de dezvoltare este cu atît mai dăunătoare
procesului de creaţie.

De aceea, în dezvoltarea acestui proiect am încercat în permanenţă să avem în
vedere toate facilităţile de care un proiectant ar avea nevoie pentru a-şi
realiza ideile sale, oferind funcţionalităţile necesare pentru a înlesni această
realizare.

\section*{Flexibilitate şi Extensibilitate}

Precum nu există o unealtă ce poate fi învăţată instantaneu, la fel nu există o
unealtă care să facă totul. Însă dorinţa noastră a fost a păstra aceste limite
superioare cît mai sus şi a oferi o gamă cît mai largă de servicii
utilizatorilor aplicaţiei de faţă.

În această privinţă, am încercat pe tot parcusul dezvoltării acestu produs ca un
viitor dezvoltator, fie el autorul sau o terţă parte, să poate porni de la o
bază solidă ce se constituie proiectul nostru şi să formeze mai departe o
unealtă mai puternică şi mai aproape de nevoile unor grupuri particulare de
utilizatori.

De departe, folosirea platformei Eclipse, despre care vom discuta în detaliu în 
cadrul Capitolului \ref{chapter:tech}, a facilitat într-o mare parte acest 
deziderat.