\chapter{Implementare}
\label{chapter:impl}

În acest capitol vom prezenta detaliile implementării acestui proiect. Se va 
trece prin structura de clase, fluxul datelor în aplicaţie, mutaţiile Modelului 
de Lucru (\ref{define:model})  şi funcţionarea componentelor de interfaţă,
editorul OpenGL şi Vizualizarea 3D, prezentate în secţiunile
\ref{section:opengl-editor}, respectiv \ref{section:view}.

\section{Structura claselor}

Fiecare dintre componentele arhitecturale descrise la capitolul 
\ref{chapter:arh} sunt materializate în implementarea efectivă printr-un set de 
clase bine definite şi bine separate ca logică şi structură de celelalte 
componente ale aplicaţiei.

\subsection{Modelul de Lucru}

\subsubsection{Primitive}
Clasa de bază a modelului de lucru este clasa Primitive. Desigur, este o
materializare a conceptului de Primitivă (\ref{define:primitive}). Aceasta este o clasă 
abstractă care nu introduce nici un comportament singular, însă marchează 
nevoie de implementare pentru intefeţele EditorRenderable şi RealRenderable, 
necesare celor două vizualizări, cum vom vedea la secţiunile 
\ref{section:impl-editor} şi \ref{section:impl-view}.

Din clasa Primitive sunt extinse clasele concrete Corner, Wall şi
toate clasele ce implementează Decoraţiunile (\ref{define:decoration}). Fiecare
din aceste clase implementează metodele definite în interfeţele amintite mai
sus.

Tot din clasa Primitive se desprinde şi clasa WallFeature, o materializare a
conceptului de Caracterstică de Ziduri (\ref{define:feature}). Din ea la rîndul
ei se concretizează clasele Door, Window, Inset, Outset, Tunnel, implementări
ale diverselor caracteristici amintite în \ref{section:primitives}.

\subsubsection{Corner}

Clasa Corner reprezintă una din puţinele Primitive care nu sunt direct
disponibile utilizatorului aplicaţiei. Ea serveşte strict poziţionării Zidurilor
şi prin aceasta editorul nu le oferă o importanţă deosebită printr-o
individualizare în cadrul interfeţei, reducînd astfel solicitarea asupra curbei
de învăţare a utilizatorului.

Clasa Corner, pe lîngă implementarea comportamentului de desenare pentru
proiectate (\ref{define:editorRender}) şi de desenare reală
(\ref{define:realRender}), aderă şi la comportamentele de Selectabilitate
(\ref{define:selectable}) şi de Navigabilitate (\ref{define:hoverable}).
