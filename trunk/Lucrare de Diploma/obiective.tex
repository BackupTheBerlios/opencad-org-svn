\chapter{Obiective}

Dintotdeauna am considerat că înainte de a iniţia orice activitate, fie că ea
este dezvoltarea unui produs software, fie că este orice altfel de sarcină de
amploare, autorul trebuie să-şi pună pe hîrtie o suită limitată de ţinte, nu
foarte uşor de realizat însă nu imposibile.

Pentru proiectul OpenCad.org, ele au fost alese prin perspectiva altor proiecte
similare dezvoltate în acest domeniu, şi am încercat să oferim ceva nou,
atractiv pentru eventualii utilizatori.

\section{Principii de proiectare}

\subsection{Uşurinţă în utilizare}

Dacă orice aplicaţie CAD tinde să ofere utilizatorilor ei unelte cît mai
puternice de dezvoltare, de multe ori se pierde din vedere ergonomia aplicaţiei.
Există un compromis ce trebuie făcut între capabilităţile aplicaţiei şi
ergonomie, însă nu credem că a cădea în oricare din tabere este benefic
rezultatului final al produsului.

De aceea, în realizarea acestui proiect am avut întotdeauna în vedere
interacţiunea cu utilizatorul şi interfaţa programului. Naturaleţea şi intuiţia
sunt cele mai bune metode de învăţare, şi nici o aplicaţie nu ar trebui să se
rezume la manualul sau documentaţia sa pentru a permite utilizatorilor să
înceapă să folosească această aplicaţie.

Noi ne-am dorit ca un utilizator să fie capabil să vadă rezultate efective
utilizînd aplicaţia noastră în cîteva minute după ce a deschis pentru prima dată
programul. Există multe aplicaţii CAD care necesită ore, dacă nu zile întregi de
acomodare cu setul de comenzi, setul de capabilităţi, de proprietăţi şi toate
celelalte componente ale unei aplicaţii de mare anvergură.

\subsection{Puternică unealtă de dezvoltare}

Desigur, nu puteam ignora scopul final al activităţii de proiectare, şi anume
redarea cît mai fidelă a ideilor proiectantului. Dacă printr-o ergonomie
defectuasă acest deziderat poate fi împedicat prin încetinirea procesului, lipsa
totală a anumitor capacităţi de dezvoltare este cu atît mai dăunătoare
procesului de creaţie.

De aceea, în dezvoltarea acestui proiect am încercat în permanenţă să avem în
vedere toate facilităţile de care un proiectant ar avea nevoie pentru a-şi
realiza ideile sale, oferind funcţionalităţile necesare pentru a înlesni această
realizare.

\subsection{Flexibilitate şi Extensibilitate}

Precum nu există o unealtă ce poate fi învăţată instantaneu, la fel nu există o
unealtă care să facă totul. Însă dorinţa noastră a fost a păstra aceste limite
superioare cît mai sus şi a oferi o gamă cît mai largă de servicii
utilizatorilor aplicaţiei de faţă.

În această privinţă, am încercat pe tot parcusul dezvoltării acestu produs ca un
viitor dezvoltator, fie el autorul sau o terţă parte, să poate porni de la o
bază solidă ce se constituie proiectul nostru şi să formeze mai departe o
unealtă mai puternică şi mai aproape de nevoile unor grupuri particulare de
utilizatori.

De departe, folosirea platformei Eclipse, despre care vom discuta în detaliu în 
cadrul Capitolului \ref{chapter:tech}, a facilitat într-o mare parte acest 
deziderat.

\section{Funcţionalitate expusă utilizatorului}

Trecînd la o prezentare mai concretă a facilităţilor aplicaţiei, vom trece în
revistă principalele funcţionalităţi de care un utilizator al aplicaţiei s-ar
putea bucura folosind acest proiect.

\subsection{Reprezentarea ideilor}

Aplicaţia de faţă trebuie să fie în primul rînd o unealtă de proiectare prin
care utilizatorul să-şi poată transpune într-un mod cît mai natural ideile sale
într-un mod fidel şi repetabil, uşor de recunoscut şi interpretat de alţi
utilizatori.

De aceea, unealta trebuie să fie în primul rînd un instrument creativ şi abia
apoi o unealtă tehnică. Spunem că utilizatorul tipic al acestui program ar vrea
să obţină din partea uneltei folosite o formă de reprezentare formală a ideilor
sale, mai presus decît un instrument tehnic de precizie.

Nu pierdem însă din vedere utilitatea practică finală a modelării, şi anume
transpunerea în activitatea de construcţie a planurilor constituite prin
intermediul acestei aplicaţii. Vom considera însă acestea secundare activităţii
de creaţie, încercînd astfel a optimiza aplicaţia pentru rapidizarea procesului
creativ mai mult decît pentru precizia inginerească.

\subsection{Disponibilitatea elementelor constructive}

Proiectantul aplicaţiilor CAD se aşteaptă ca programul să dispună de o serie de
elemente constitutive ale proiectelor sale comune unui subset de reprezentări
ale construcţiilor efective.

Considerăm disponibilitatea a diverselor modele cu care poate fi completat
proiectul drept esenţială transpunerii cît mai fidele a ideilor proiectantului.

În acelaşi timp însă, considerăm importantă separarea de natura şi parametrii
diverselor elemente constituente secundare, cum ar fi de exemplu
profilul cadrului unei ferestre, forma mînerului sau materialele din care acea
fereastră ar fi construită.

Detaliile cu privinţă la materiale şi tipologie a elementelor constructive
constituie ultima prioritate a acestui proiect, tocmai pentru că s-a dorit ca
proiectantul să poată să se concentreze absolut pe nevoile sale creative decît
pe detaliile ce sunt mai puţin relevante finalizării proiectului său.

\subsection{Separarea sarcinilor de proiectare de explorarea modelului}

Deşi este nevoie de o îmbinare perfectă între modelul la nivelul logic şi
reprezentarea sa într-o lume tridimensională, adeseori îmbinarea activităţii de
proiectare cu o reprezentare tridimensională duce la îngreunarea interacţiunii
cu modelul.

De aceea, sarcina de proiectare pentru utilizatorul proiectului nostru va fi
complet disjunctă de sarcina de explorare a modelului în reprezentarea sa reală.
Îndeplinirea acestui deziderat va duce la îmbunătăţirea timpilor de lucru şi a
curbei de învăţare, datorită faptului că instrumentele plane sunt mult mai uşor
de înţeles şi de folosit pe termen scurt şi apoi de refolosit pe termen lung
decît diverse instrumente tridimensionale.

\subsection{Coerenţa în utilizare}

O sarcină poate fi realizată într-un singur fel. Acest principiu este unul
simplu, care poate fi disputat asupra eficienţei sale, însă cu siguranţă oferă o
claritate greu de înlocuit de alte forme de proiectare.

Interfeţele grafice ale programelor CAD de multe ori tind să ofere o multitudine
de facilităţi şi, din raţiuni comerciale sau istorice, multe facilităţi sunt de
fapt variante subtile ale altor facilităţi existente.

Coerenţa este un puternic instrument de simplifcare a interfeţei cu
utilizatorul. De asemenea, reduce cu mult volumul de lucru şi timpii de
realizare prin formarea timpurie a reflexelor de proiectare (i.e. Cum se face
asta? -- cu cît răspunsul este mai simplu cu atît el va fi reţinut mai uşor)

\subsection{Disponibilitatea şi Organizarea datelor}

Utilizatorul va avea nevoie de informaţia pe care el a introdus-o în program în
diverse formate. Aplicaţie trebuie să suporte transportul la distanţă a
informaţiei, colaborarea între diverse echipe de proiectanţi, ş.a.m.d.

Ar trebui avută în vedere, de asemenea, posibilitatea exportului datelor în alte
programe şi interacţiunea cu modele realizate cu alte unelte, în limita
posibilităţilor.

De asemenea, programul trebuie să fie capabil să păstreze şi să modularizeze 
creaţia utilizatorului, în proiecte, dosare şi fişiere. O etapă importantă în 
lucrul la proiecte mai ample o reprezintă şi implementarea facilităţilor de 
modularizare a diverselor componente în componente mai complexe.

Aplicaţia trebuie să ofere de asemenea acces la metode moderne de stocare a
datelor pe Internet, pe servere ftp sau http, sisteme de control al versiunilor,
etc.

\subsection{Standardizare şi interoperabilitate}

Gama utilizatorilor astfel de aplicaţii este largă şi de multe ori aceştia
provin din medii diferite de educaţie şi cultură. De aceea, este de o mare
imporanţă folosirea reprezentărilor standardizate pentru diverse elemente ce vor
fi folosite în proiectare.

De la unitatea de măsură folosită, la dimensiunile standard pentru diverse
obiecte, la reprezentările schematice ale acelor elemente, toate aceste aspecte
trebuie avute în vedere. Standardizarea oferă asemenea altor caracteristici
enunţate mai sus uşurinţă în adaptarea şi aderarea la utilzarea aplicaţiei pe
termen lung de către proiectanţi.

Standardizarea oferă de asemenea şansa de extinderii rapide a ariei de utilizare
a acestui tip de aplicaţii, şi de folosirea modelelor create în diverse etape
ale fazei de proiectare şi construcţie.

\section{Considerente de performanţă}

Am considerat în general că limitarea capacităţilor unei aplicaţii datorită
performanţelor pe care poate să le realizeze duce în final la limitarea
capacităţii creative a utilizatorului.

Am încercat să identificăm principalele probleme de perfomanţă care ar putea
dăuna cel mai mult în sensul exprimat mai sus.

\subsection{Scalarea la dimensiunea datelor de intrare}

Aplicaţia trebuie să fie slab sensibilă la creşterea volumului de date şi a
complexităţii modelelor realizate. De multe ori, activitatea de proiectare poate
merge de la modele foarte simple pînă la ample proiecte arhitectonice, toate ar
trebuie să fie realizabile cu aceeaşi performanţă a aplicaţiei.

Această caracteristică asigură de asemenea stabilitatea sistemului în condiţii
de solicitare la volum mare de date. De multe ori o aplicaţie poate avea
probleme cu utilizarea memoriei sau a spaţiului pe disc dacă este puternic
sensibilă la volumul datelor procesate, chiar dacă comportamentul ei la un volum
redus de date este rezonabil.

\subsection{Solicitarea şi dependinţa de soluţia grafică disponibilă}

Este de inevitabil într-o aplicaţie de această natură folosirea intensivă a
facilităţilor grafice ale clientului pe care se rulează aplicaţia. Însă există
numeroşi factori prin care o aplicaţie îşi poate optimiza această solicitare a
sistemului grafic al gazdei.

Am avut în vedere în realizarea acestei aplicaţii optimizarea sarcinilor ce
necesită folosirea sistemului grafic, şi în acelaşi timp, prin modelarea
reprezentărilor entităţilor s-a încercat limitarea complexităţii lor geoemtrice,
fără a pierde bineînţeles din sugestivitatea formelor.

\subsection{Dependinţa redusă de resursele generale ale sistemului}

Utilizatorii acestei aplicaţii nu sunt neaparat posesorii unor soluţii de calcul
de ultimă generaţie. De aceea, aplicaţia ar trebui să se scaleze destul de lejer
şi în jos spre diverse arhitecturi de calcul mai slab performante.

Dacă progresul tehnologiei sistemelor de calcul a permis avansarea aplicaţiilor
software spre limite greu de imaginat cu ani în urmă. Aceasta s-a întîmplat mai
ales în sfera aplicaţiilor grafice. Nu trebuie însă să pierdem din vedere şi
soluţiile performante ale trecutului, care oferă suficiente resurse pentru
rularea în condiţii optime a unor soluţii software de mare complexitate.

\section{Considerente de implementare}

\subsection{Stabilitate. Securitarea şi Integritatea datelor}

Nimic nu este mai perturbant activităţii de proiectare decît o unealtă care
funcţionează greşit sau care nu funcţionează deloc, care are tendinţa să piardă
date şi să se defecteze exact în momentele cele mai importante.

Nu trebuie pierdut deloc din vedere (mai ales cînd mare parte din aceste sarcini
pot fi realizate automat în cadrul aplicaţiilor moderne) fluxul datelor şi modul
în care se poate preveni pierderea de informaţii în cazuri extreme de
funcţionare a sistemului de calcul sau a diverselor erori accidentale ce apar în
cadrul execuţiei programului.

De asemenea, programul trebuie să încerce să expună cît mai puţine
vulnerabilităţi în ceea ce priveşte pierderea de informaţii prin acţiuni
controlate împotriva aplicaţiei menite să cauzeze perturbarea funcţionării şi
integrităţii datelor în cadrul aplicaţiei.

\subsection{Portabilitate şi independenţa de platformă}

Aplicaţiile moderne nu mai sunt atît de ancorate faţă de sistemul de calcul şi
sistemul de operare care îl ghidează. În acest spirit, aplicaţia de faţă va
putea să fie executată pe diverse sisteme de operare, oferind de asemenea
independenţă şi faţă de sistemul hardware ce susţine tehnologiile implicate în
dezvoltarea acestui proiect.
