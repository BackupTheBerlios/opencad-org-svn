\chapter{Tehnologii}

\section{Eclipse Rich Client Platform}

Proiectul Eclipse îşi are începuturile alături de compania IBM de care s-a 
separat în 2003; astăzi este printre cele mai cunoscute proiecte opensource, 
fiind recunoscut ca un foarte popular mediu de dezvoltare de aplicaţii Java şi 
nu numai.

Binecunoscutul mediu de dezvoltare Eclipse este dezvoltat pe baza Eclipse Rich 
Client Platform. Aceasta reprezintă suita minimă de plugin-uri necesare pentru 
dezvoltarea unei aplicaţii "rich client" \footnote{eng. \textit{client bogat} 
-- O aplicaţie desktop care beneficiază de un set de elemente de interfaţă 
evoluate ("bogate"), care oferă o putere mare de abstractizare şi în acelaşi 
timp control al funcţionalităţii programatorului, păstrînd o experienţă de 
utilizare plăcută pentru utilizator. În general orice aplicaţie desktop care 
foloseşte elemente de interfaţă predefinite pînă la un anumit punct poate fi 
considerată ca un rich client.}. Deşi platforma este destinată dezvoltării de 
aplicaţii tip IDE, prin înlăturarea anumitor componente ale platformei se pot 
dezvolta orice tip de aplicaţii desktop.

Structura de bază a acestei platforme poate fi rezumată după cum am prezentat 
în Tabela \ref{table:rcpstruct}. \begin{table}[h]
\caption{Structura Eclipse RCP \cite{rcpfaq} \label{table:rcpstruct}}
\begin{tabular}{|\col{0.23}|\col{0.73}|}
\hline Eclipse Runtime & Suport pentru plugin-uri, puncte de extensie şi 
extensii. Construită peste framework-ul OSGi\\
\hline SWT & Proiectat să ofere acces eficient şi portabil la facilităţile de 
interfaţă utilizator ale sistemului de operare pe care este implementat\\
\hline JFace & Un framework de interfaţă utilizator, construit pest SWT, pentru 
tratarea multor sarcini de programare de interfeţe\\
\hline Workbench & Construit peste toate componentele anterioare, aduce un 
mediu de lucru ultra scalabil, cu interfaţă publică ce suportă mai multe 
ferestre pentru administrarea vizualizărilor, editoarelor, perspectivelor, 
acţiunilor, preferinţelor şi multe altele. \\
\hline
\end{tabular}
\end{table}

\subsection{Eclipse Runtime}
Această componentă a Eclipse RCP este cea mai abstractă şi în acelaşi timp cea 
mai puternică componentă a sa. Structura sa de bază este descrisă în Tabela 
\ref{table:runtimestruct}.

\begin{table}[h]
\caption{Structura Eclipse Runtime \cite{eclipsehelp} \label{table:runtimestruct}}
\begin{tabular}{|\col{0.38}|\col{0.58}|}
\hline org.eclipse.core.contenttype & Suport pentru definirea şi administrarea 
tipurilor de fişiere \\
\hline org.eclipse.core.jobs & Infrastructură pentru programare paralelă în 
Eclipse \\
\hline org.eclipse.equinox.registry & Sistemul prin care un plugin publică alte 
plugin-uri de care depinde şi defineşte puncte de extensie pentru ca alte 
plugin-uri să-i îmbogăţească funcţionalitatea \\
\hline org.eclipse.equinox.preferences & O infrastructură prin care un plugin 
îşi păstrează preferinţele în seturi de perechi cheie/valoare modificabile de 
către utilizator \\
\hline org.eclipse.equinox.common & Administrarea proiectelor, resurselor şi 
fişierelor \\
\hline
\end{tabular}
\end{table}

Equinox este o implementare a specificaţiilor OSGi R4
\footnote{\url{http://osgi.org/osgi_technology/download_specs.asp?section=2}}
o serie de plugin-uri care implementează diverse servicii opţionale OSGi şi 
alte infrastructuri pentru rularea sistemelor bazate pe OSGi. \cite{equinox}

Equinox produce, printre alte lucruri, implementarea OSGi folosită de Eclipse 
RCP (şi celelalte proiecte bazate pe Eclipse) ca un model de componente.
\cite{equinoxfaq}

\subsection{SWT}
SWT\footnote{abrev. \textit{Standard Widget Toolkit}} este una din cele mai 
populare tehnologii dezvoltate de Fundaţia Eclipse. Această tehnologie 
foloseşte la crearea de interfeţe utilizator în mediul de dezvoltare Java, 
oferind programatorului capacitatea de a construi interfeţe utilizator 
portabile între diverse sisteme de operare. În prezent suportă toate sistemele 
de operare uzuale şi diverse arhitecturi maşină (e.g. x86, amd64).

Structura sa este similară cu cea a altor toolkit-uri similare, ca AWT 
\footnote{abrev. \textit{Abstract Window Toolkit} -- toolkit-ul IU standard 
Java \url{http://en.wikipedia.org/wiki/Abstract_Windowing_Toolkit}} şi 
SWING\footnote{Un toolkit IU independent de platformă 
\url{http://en.wikipedia.org/wiki/Swing_(Java)}}. Elementele fundamentale sunt 
Shell-ul şi Controlul. Un shell este abstracţia conceptului de fereastră şi 
reprezintă în general un container pentru controale. Controalele sunt 
elementele interfeţei utilizator. Butoanele, etichetele, arborii şi tabelele 
sunt toate controale şi utilizatorii sunt obişnuite cu ele din alte programe 
ale desktopului. \cite{swt} În principiu orice widget poate fi simplu sau 
compus (i.e. descendent al clasei Composite) şi întreaga lor ierharhie este 
controlată de relaţiile dintre clasele SWT.

Interacţiunile dintre controale şi utilizator cît şi controlul tranziţiei 
stărilor controalelor se face cu ajutorul Evenimentelor. Fiecare modificare de 
stare a unui control (cum ar fi apăsarea pe un buton) este semnalată de SWT 
către aplicaţie prin declanşarea unui eveniment. Orice Listener (ascultător 
înregistrat pentru un anumit eveniment) va primi o notificare (i.e. o metodă 
declarată în interfaţa de Listener va fi apelată) ce va conţine evenimentul ce 
a avut loc. Orice clasă poate deveni listener prin implementarea unor interfeţe 
specifice, SWT oferind astfel o mare flexibilitate în cadrul aplicaţiilor care 
îl folosesc.

Ca şi în AWT sau SWING, poziţia controalelor este stabilită cu ajutorul 
Layout-urilor. Un layout reprezintă un set de reguli de poziţionare pentru un 
control compus ce va decide pentru toate controalele incluse cum vor fi ele 
poziţionate pe ecran. Unele layout-uri asociază anumite date fiecărui control 
pentru a reţine poziţia în care va fi desenat. Poziţionarea manuală a tuturor 
controalelor este o problemă complexă şi dacă acel algoritm nu este scris 
eficient, el poate afecta performaneţele întregii aplicaţii. De aceea, SWT 
oferă nişte seturi predefinite de layout-uri ce serversc cele mai comune cazuri 
întîlnite de programatori.

\subsection{JFace}
JFace este un toolkit pentru interfeţe utilizator ce conţine clase pentru 
tratarea multora din sarcinile uzuale de programarea interfeţelor utilizator. 
JFace este independent de sistemul de ferestre atît in interfaţa sa pentru 
programatori cît şi în implementare, şi este proiectat să funcţioneze cu SWT 
fără a-i ascunde funcţionalitatea.

JFace include componentele uzuale de toolkit IU, regiştri de imagine şi seturi 
de caractere, text, dialog-uri, framework-uri pentru preferinţe şi wizard-uri
\footnote{Dialogurile des întîlnite ce automatizează diverse sarcini repetitive}
şi raportarea progresului pentru sarcini cu durată mare de execuţie.
\cite{jface}

Pachete principale din JFace, care oferă funcţionalitatea de bază a tehnologiei 
sunt prezentate în Tabela \ref{table:jfacestruct}

\begin{table}[htb]
\caption{Structura JFace \cite{jface-art} \label{table:jfacestruct}}
\begin{tabular}{|\col{0.13}|\col{0.28}|\col{0.53}|}
\hline Ferestre & org.eclipse.jface.window & Facilităţi de creerea şi 
administrarea ferestrelor. Interesantă este clasa ApplicationWindow, care aduce 
un nou nivel de abstracţie peste conceptul de fereastră şi înglobează o buclă 
program SWT.\\
\hline Vizualizări & org.eclipse.jface.viewers & Vizualizări ca şi TreeViewer 
sau TableViewer, care sunt componente ghidate de model ce folosesc widgeturi 
SWT şi adaptează conţinutul modelului la conţinutul widgetului.\\
\hline Dialog-uri & org.eclipse.jface.dialogs & Diverse dialog-uri des 
folosite.\\
\hline Acţiuni & org.eclipse.jface.actions & Un framework de acţiuni IU similar 
cu cel găsit în SWING pentru a implementa comportament partajat între două sau 
mai multe componente, cum ar fi o intrare în meniu şi un buton de pe bara de 
instrumente.\\
\hline Wizard-uri & org.eclipse.jface.wizard & Un framework avansat de 
construcţie a wizard-urilor.\\
\hline Resurse & org.eclipse.jface.resource & Suport pentru administrarea 
resurselor, cum ar fi imaginile şi fonturile SWT.\\
\hline Text & org.eclipse.jface.text & Un framework de crearea, manipularea, 
afişarea şi editarea documentelor text.\\
\hline
\end{tabular}
\end{table}

\subsection{Workbench}
Această parte a Eclipse RCP reprezintă o colecţie largă de clase şi interfeţe 
pentru construirea de interfeţe utilizator complexe.

Iată principalele elemente ce constituie un Workbench.

\subsubsection{Workbench}
Ca termen strict, se referă la fereastra care conţine întreaga aplicaţie. 
Defineşte un container abstract pentru toate elementele de interfaţă din 
aplicaţia dezvoltată cu Eclipse RCP.

\subsubsection{Pagina}
Reprezintă, în mod intuitiv,  toată partea ferestrei care nu conţine barele de 
unelte şi meniul.

\subsubsection{Perspective}
Folosesc pentru organizarea conţinutului într-o Pagină. O perspectivă defineşte 
o colecţie de Vizualizări, aşezarea lor şi acţiunile ce pot fi efectuate pentru 
o sarcină a utilizatorului. Perspectivele pot fi schimbate de către 
utilizatorul. Modificarea Perspectivei afectează doar Vizualizările nu şi 
Editoarele. \cite{eclipsehelp}

\subsubsection{Vizualizările şi Editoarele}
Reprezintă extensiile aduse de programator într-o aplicaţie Eclipse RCP. Un 
editor urmăreşte modelul unei maşini de stări în sensul că el are o etapă de 
deschidere, editare şi o etapă de salvare.

În schimb orice modificare a stării unei Vizualizări este salvată imediat. 
Vizualizările sunt folosite la a arăta structura modelului deschis într-un 
Editor, la a facilita accesul la fişiere şi la alte resurse ale unui proiect.

Editoarele urmăresc în general modelul unui editor de text clasic şi oferă 
aceeaşi funcţionalitate însă la un nivel mai abstract, nefiind dependent de o 
intrare ca fişier text, nici măcar ca intrarea să fie orice fel de fişier.

\section{OpenGL şi extensia OpenGL pentru SWT}

OpenGL\footnote{abrev. \textit{Open Graphics Library}} este o specificaţie de 
standard independent de limbaj ce definieşte un API pentru scrierea de 
aplicaţii ce produc grafice tridimensionale sau bidimensionale pe calculator. 
Interfaţa este compusă din peste 250 de apeluri de funcţii ce pot fi folosite 
pentru a desena scene tridimensionale complexe din primitive simple. OpenGL a 
fost dezvoltat de SGI\footnote{abrev. \textit{Silicon Graphics Inc.}} în 1992. 
Este folosit pe scară largă în aplicaţii CAD, realitate virtuală, vizualizare 
ştiinţifică, vizualizarea informaţiei, simularea zborului, dezvoltare de jocuri 
pe calculator, ş.a. \cite{oglwiki}

\subsection{OpenGL în Java}
Java reprezintă una dintre cele mai puternice platforme de dezvoltare de 
aplicaţii cunoscute în ziua de azi. Dezvoltată iniţial pentru a separa sarcina 
de dezvoltare de aplicaţii de platforma pe care va rula, astăzi limbajul Java 
este folosit în numeroase aplicaţii software, începînd de la clienţi destop 
pînă la jocuri pentru telefoane mobile şi aplicaţii web. Un astfel de sistem nu 
putea să rămînă fără o implementare a OpenGL. În prezent, există două extensii 
OpenGL pentru Java, JOGL şi LWJGL.

JOGL\footnote{Java OpenGL Library \url{http://jogl.dev.java.net}} este varianta 
agreată de Sun şi urmează să facă parte din distribuţia oficială a 
JDK\footnote{După cum precizează JSR231 
\url{http://jcp.org/en/jsr/detail?id=231}}. Principala sa caracteristică este 
că foloseşte AWT ca şi suport de interfaţă grafică.

LWJGL\footnote{Lightweight Java Game Library \url{http://lwjgl.org/}} este o 
soluţie mai amplă dedicată dezvoltării de jocuri în OpenGL. Librăria oferă şi 
suport pentru tratarea sunetului şi a evenimentelor de interacţiune cu 
utilizatorul, specializate cum am spus după nevoile programării jocurilor pe 
calculator.

\subsection{OpenGL pentru SWT}

Din păcate, nici una din aceste tehnologii nu se împacă foarte bine cu SWT. 
Pentru că atît OpenGL cît şi SWT şi AWT implică apeluri native către sistemul 
de operare, de multe ori aceste apeluri sunt ignorante unele de celelalte şi 
multe interferenţe apar între interfaţa grafică desenată cu SWT şi fereastra de 
desenare OpenGL.

Din aceste raţiuni, o a treia variantă a fost aleasă de noi, anume extensia 
OpenGL pentru Java făcută de Eclipse pentru SWT. Este de menţionat că rezultate 
similare s-au obţinut şi LWJGL, care la rîndul său poate fi integrat cu puţin 
efort cu SWT. JOGL din păcate, datorită legăturii sale puternice cu AWT, nu 
poate fi adus în scopul aplicaţiei noastre într-un mod care să poate fi uşor 
reproductibil de autor.

Limitările bibliotecii OpenGL pentru SWT făcute de Eclipse se referă la 
versiunea de OpenGL suportată (i.e. OpenGL 1.1) şi la faptul că interesul 
pentru această tehnologie este oarecum scăzut, ultima versiune existentă 
datează din septembrie 2005\footnote{\url{http://www.eclipse.org/swt/opengl/}} 
şi sunt şanse ca versiuni ulterioare să nu mai existe, pe cînd celelalte două 
librării sunt în continuă dezvoltare.

La momentul determinării acestei alegeri, după o testare a tuturor 
tehnologiilor disponibile, am ajuns la concluzia că funcţionalitatea oferită de 
varianta Eclipse a legăturii cu OpenGL este atît suficientă ca şi capaciţăţi 
necesitate de acest proiect cît şi destul de stabilă pentru a oferi o calitate 
înaltă de producţie.

\section{Raţiunea alegerii tehnologiilor}

Încă de la începutul dezvoltării acestui proiect, s-a pus problema tehnologiei 
care să fie folosită pentru îndeplinirea cît mai facilă a obiectivelor 
proiectului. Munca de cercetare în privinţa alegerii tehnologiei potrivite a 
implicat o documentare amplă asupra tehnologiilor luate în considerarea cît şi 
testarea practică a multora dintre ele.

Vom încerca mai jos să trecem în revistă o serie de alegeri care le-am făcut şi 
raţiunea ce stă în spatele lor, prin prisma avantajelor şi dezavantajelor aduse 
de alegerile făcute.

\subsection{PHP şi Python}

Consideraţia asupra alegerii limbajului de programare în care urma să fie 
dezvoltat acest proiect a început desigur de la aria cunoştinţelor autorului. 
Alegerea s-a făcut astfel între C, C++ şi Java. Alte alternative mai 
,,exotice'' au fost luate în considerare marginal, cum ar fi 
Python\footnote{\url{http://www.python.org/}} sau 
PHP\footnote{\url{http://www.php.net/}}.

Capacităţile limbajelor scripturale de a intercaţiona cu libării native şi de a 
crea interfeţe grafice a cunoscut o dezvoltare masivă în ultima perioadă. 
Există pentru Python PyGTK\footnote{\url{http://www.pygtk.org/}}, o librărie ce 
permite programarea facilă a intefeţelor grafice cu GTK\footnote{abrev. 
\textit{The GIMP Toolkit} \url{http://www.gtk.org/}} sub Python. O extensie 
similară există pentru PHP, numită PHP-GTK\footnote{\url{http://gtk.php.net/}}.

Din punctul de vedere al graficii tridimensionale, extensii similare există 
pentru ambele limbaje, oferind funcţionalitate similară cu extensia prezenză în 
limbajul Java.

La nivel de dezvoltarea de aplicaţii desktop complexe, pentru Python există 
framework-ul DABO\footnote{\url{http://dabodev.com/} Un framework orientat pe o 
funcţionalitate client similară aplicaţiei Visual FoxPro} bazat pe 
wxPython\footnote{\url{http://www.wxpython.org/} un framework similar cu 
wxWidgets (despre care vom discuta ulterior), ce oferă portabilitate pentru 
codul intefeţei grafice între diverse sisteme de operare}. Apropierea acestor 
sisteme de cerinţele noastre nu a fost evidentă, cît şi popularitatea relativă 
a acestor proiecte a făcut considerarea lor doar trecătoare.

\subsection{C şi C++}

Revenind la principalii candidaţi, limbajele tradiţionale pentru dezvoltarea de 
aplicaţii software nu puteau să treacă neobservate autorului. Există o suită 
impresionantă de aplicaţii desktop dezvoltate exclusiv doar pe baza acestor 
două limbaje de operare.

Ca şi set de widgeturi interesante pentru noi, menţionăm librăriile GTK, 
QT\footnote{\url{http://en.wikipedia.org/wiki/Qt_(toolkit)}} şi bineînţeles 
.NET Framework\footnote{\url{http://msdn.microsoft.com/netframework/}}.

Programarea la nivel de bază, cum este cea în limbajul C nu a fost niciodată 
agreată de autor pentru capacităţile limitate de scalare (sau viteza redusă de 
scalare şi efortul depus pentru scalare), lipsa unei integrări independente de 
platformă la un nivel superior şi cu o structură coerentă, orientată obiect.

Aceste considerente au cîntărit greu împotriva bibiliotecii GTK, cea mai 
populară arhitectură open-source de dezvoltare de aplicaţii grafice în C, ce 
stă la baza sistemului desktop GNOME\footnote{\url{http://www.gnome.org/}, un 
sistem destkop complet, întîlnit în majoritatea distribuţiilor Linux cît şi a 
majorităţii altor platforme UNIX}.

În contrabalans a stat librăria gtkmm\footnote{\url{http://www.gtkmm.org/}}, o 
extensie a lui gtk pentru C++. Pînă în ziua de astăzi, o despărţire clară de 
această tehnologie din partea autorului nu există. Unicele considerente 
împotriva acestei alegeri rămîn dependinţa de limbajul C++ şi lipsa unei 
portabilităţi între sisteme de operare de o stabilitate apreciabilă.

Răspunsul acestor lipsuri a părut să îl aibă 
wxWidgets\footnote{\url{http://www.wxwidgets.org/}}. O platformă cu o tradiţie 
considerabilă în dezvoltarea de aplicaţii client, wxWidgets reprezintă poate 
cea mai bună alegere pentru un dezvoltator de aplicaţii client portabile sub 
C++, în opinia noastră.

Însă o analiză mai profundă a acestei platforme a scos la iveală faptul că 
wxWidgets există ca şi sistem de dezvoltare de aplicaţii client dinaintea 
extensiei STL\footnote{abrev. \textit{Standard Template Library} 
\url{http://en.wikipedia.org/wiki/Standard_Template_Library}} pentru C++. De 
departe -- în opinia autorului -- cea mai atractivă componentă a limbajului C++ 
suferă de un suport consistent pentru wxWidgets.\cite{wxfaq} Această 
caracteristică a bibliotecii a reprezentat un criteriu de nedepăşit în alegerea 
noastră.

Biblioteca QT a trebuit să fie dată la o parte pe baza principiului 
portabilităţii. Din păcate, există puţine implementări de succes de software 
implementat cu QT pentru platforma Windows. De asemenea, QT nu este distribuit 
sub o licenţă compatibilă cu intenţiile de distribuţie ale autorului.

\subsection{.net Framework}

Dacă pînă acum ne-am uitat la tehnologii care pornesc dinspre domeniul 
open-source şi Linux, .net Framework reprezintă răspunsul gigantului Microsoft 
la evoluţia sistemelor software moderne.

Cu o structură de clase avansată, .net Framework este printre puţinele soluţii 
software atît de impresionante care nu au reuşit să-şi găsească un răspuns 
rezonabil în lumea opensource.

Deşi reprezintă ţelul unui întreg proiect, 
mono\footnote{\url{http://www.mono-project.com}}, la data la care am făcut 
cercetările noastre, portabilitatea sistemului .net pentru platforme UNIX nu a 
ajunsese încă, în opinia autorului, la o maturitate demnă de luat în seamă. 
Aflat într-o continuă dezvoltare, proiectul mono promite să realizeze o 
revoluţie în ceea ce priveşte portabilitatea programelor software pe diverse 
sisteme de operare.

\subsection{Java}

Dintre toate sistemele de dezvoltare software existente astăzi în lumea 
dezvoltatorilor de aplicaţii desktop, noi am decis că Java reprezintă cel mai 
convenabil, uşor de dezvoltat, uşor de menţinut şi uşor de distribuit mod de a 
dezvolta proiectul de faţă.

Cu o istorie legendară de realizări, Java reprezintă poate cel mai dezvoltat 
sistem software existent, cu biblioteci dezvoltate de un număr impresionant de 
programatori din întreaga lume, cu o serie de tehnologii pentru toate 
aplicaţiile software cunscute.

Mai mult, deschiderea perspectivelor software către alte platforme decît cele 
tradiţionale (i.e. Mac, PC), cu dezvoltarea din ce în ce mai mare a 
tehnologiilor web şi a nevoilor crescînde în domeniul server, Java tinde să 
devină cel mai variat sistem software existent vreodată. În ceea ce priveşte 
dezvoltarea aplicaţiilor client, Java a cunoscut o serie de etape remarcabile 
pe parcursul evoluţiei sale. Primul sistem de interfeţe pentru utilizatori a 
fost AWT, tehnologie care permitea legarea apelurilor native de construcţie a 
widget-urile grafice de o structură de clase abstractă şi independentă de 
platforma de execuţie. Dezvoltarea tehnologiei a fost însă prematur întreruptă, 
fiind folosită astăzi doar sporadic în dezvoltarea de aplicaţii client, acolo 
unde performanţa este considerată prioritară uşurinţei de utilizare a 
aplicaţiei.

A urmat SWING, care reprezintă răspunsul dezvoltatorilor Java la problemele 
utilizatorilor cu platforma AWT. Rămînînd complet în spectrul Java, codul 
bibliotecii SWING nu interacţionează direct cu sistemul de operare pe care 
rulează aplicaţia ci alege să-şi deseneze întreaga interfaţă folosind libăria 
Java 2D\footnote{\url{http://java.sun.com/products/java-media/2D/}}. Pierzînd 
contactul cu interfaţa nativă, SWING a avut dintotdeauna problema performanţei. 
Evoluţia tehnologiei a adus progrese semnificative în acest sens. SWING aducea 
de asemenea o nouă formă a interfeţei grafice, diferită de orice ce altă 
interfaţă a sistemelor de operare.

Acest aspect este considerat de mulţi un atu al tehnologiei, permiţînd 
aplicaţiei să arate la fel pe sisteme de operare diferite. Neajunsul este însă 
faptul că utilizatorii noi trebuie să se obişnuiască cu o nouă interfaţă 
grafică cînd deschid aplicaţia, lucru ce poate afecta prima impresie şi curba 
de învăţare a utilizatorilor.

Eclipse a început proiectul SWT pentru dezvoltarea aplicaţiei Eclipse. 
Proiectul dorea să ofere tot ce AWT nu a reuşit să ofere. Printre 
caractersticile principale ale SWT este că foloseşte un număr mai mare de 
widget-uri comune decît AWT şi că pentru platformele UNIX foloseşte biblioteca 
gtk de widget-uri, faţă de biblioteca 
motif\footnote{\url{http://www.opengroup.org/motif/}} folosită de AWT, o 
tehnologie care astăzi a fost lăsată în urmă pentru dezvoltarea de interfeţe 
grafice în sisteme UNIX/Linux.
\subsection{Eclipse RCP}

În final, vom încerca să motivăm şi alegerea folosirii RCP pentru dezvoltarea 
acestui proiect. Considerentele de bază au pornit în primul rînd de la succesul 
aplicaţiei Eclipse ca şi un IDE pentru documente Java, aplicaţii web, ş.a. 
Cunoaşterea şi familiarizarea cu modul de lucru al acestui IDE au dus 
inevitabil la recunoaşterea calităţilor de interfaţă utilizator ale Eclipse 
RCP, care stă la bază Eclipse IDE.

Deşi RCP a fost proiectat în principal pentru editarea documentelor de tip 
text, am considerat că experimentarea cu un nou tip de model de editare, cea de 
modelare tridimensională poate deschide o nouă perspectivă în aria de aplicaţii 
a Eclipse RCP.

După iniţierea dezvoltării proiectului, mare parte a premizelor asupra 
uşurinţei de extindere şi dezvoltare a platformei s-au confirmat. Bineînţeles, 
volumul cel mai mare de muncă l-a cerut ,,învăţarea'' lui Eclipse să editeze un 
model tridimensional, sarcină care la rîndul ei a fost facilă datorită 
abstracţiei puternice de care se bucură structura aplicaţiilor dezvoltate în 
RCP.
