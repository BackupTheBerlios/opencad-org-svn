\chapter{Concluzii}

Vom încerca să prezentăm succint, în această ultimă parte, o serie de concluzii
şi observaţii la care am ajuns în procesul de dezvoltare a acestui proiect şi în
momentul finalizării sale.

În primul rînd, trebuie să subliniem rolul covîrşitor pe care l-a avut Eclipse
Rich Client Platform (\ref{section:rcp}) în acest proiect. Aşa cum ne aşteptam,
platforma în cauză reprezintă o bază solidă pentru dezvoltarea aplicaţiilor
desktop şi, în ciuda specializării sale pentru unelte de dezvoltare IDE pentru
programare, s-a dovedit a fi destul de generică pentru a oferi suport pentru
tipul de aplicaţie care l-am avut în vedere.

Integrarea Eclipse RCP şi a OpenGL a funcţionat la nivelul aşteptărilor, oferind
o bază reală de dezvoltare, identică -- dacă nu mai puternică -- cu alte soluţii
pentru grafica tridimensională. Plugin-ul pentru OpenGL dezvoltat de Fundaţia
Eclipse reprezintă deci un început bun pentru aducerea capacităţilor 3D în
aplicaţii dezvoltate cu şi pentru Eclipse.

Considerăm că aplicaţia de faţă a reuşit să întîlnească prospectele iniţiale în
ceea ce priveşte uşurinţa de folosire şi integrarea cu alte unelte Eclipse.
Funcţionalităţile implementate pot fi duse mai departe prin puterea de extensie
a platformei şi prin nivelul de abstracţie la care acest proiect a fost
proiectat. 

În ceea ce priveşte utilitatea aplicaţiei, credem că poate reprezenta un punct
de pornire pentru o unealtă de proiectare de nivelul altor astfel de aplicaţii
consacrate, şi poate fi folosită şi în forma sa actuală de către proiectanţii de
ansambluri arhitectonice.

Pentru viitor, există posibilitatea dezvoltării în continuare a acestei soluţii,
oferind astfel din ce în ce mai multe facilităţi şi o interfaţă mai puternică
pentru viitorii utilizatori.
